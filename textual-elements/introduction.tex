\chapter[Introduction]{Introduction}

With the advance of technology, music - and musical instruments - have also evolved to
use it's advantages. They are a lot of use cases, the most noticeable ones being music
annotation and creation (through electronic instruments, also known as synthesizers).
All of the modern musical software and hardware use the same format to communicate,
called MIDI. \\
To translate music playing it is needed to know which note is being played at a
given time. This make it very easy to translate instruments that have separated
keys for each note (the most noticeable one being piano) to MIDI, but very hard
do the same for instruments like the guitar, that have a single output for multiple
notes (each string has about 15 notes). The guitar is also a harmonic instrument,
as it can play multiple notes simultaneously, which makes it's digitization even harder.\\
There are already a few commercial solutions for this, but not a very performant
and cheap one. Recently a new pure software solution was released at a reasonable
price which works very well for live MIDI playing, but not enough for music annotation.
There are also a few hardware solutions available at the market, which perform well,
but are very expensive. \autoref{market-solutions} shows the most relevant solutions
in the current market.

\begin{table}[htb]
  \begin{center}
    \ABNTEXreducedfont
    \caption[Market Solutions]{Market Solutions}
    \label{market-solutions}
    \begin{tabular}{m{4cm} | m{1cm} | m{2cm} | m{2cm} | m{2cm} }
      \hline
      Name & Price (U\$D) & \pbox{2cm}{Usage\\Complexity} & \pbox{2cm}{Live\\Performance} & \pbox{2cm}{Annotation\\Performance}\\
      \hline \hline
      Roland GK3 + GhostHexpander + GI20 & 700 & Hard & High & High \\
      Godin Freeway + GhostHexpander + GI20 & 800 & Hard & High & High \\
      Jam Origin - Audio to MIDI & 100 & Easy & High & Low/Medium \\
      Migic & 40 & Easy & Medium & Low \\
      \hline
    \end{tabular}
    \legend{Source: authors}
  \end{center}
\end{table}
