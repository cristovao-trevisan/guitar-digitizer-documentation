\chapter{Tools Selection}
\label{tools-selection}
The firmware is basically an analog sampler, all it has to do is sample six analog
channels, add a header (to identify the beginning and check continuity) and send it
through USB. Lets start by listing the requirements for the hardware.

\section{Requirements}
\begin{itemlist}
  \item Super cheap
  \item 6 analog channels (more precision is better)
  \item High sample rate (at least 10 kHz for each channel, but ideally 44 kHz or more)
  \item Fast USB support, to send the data with headers
\end{itemlist}
Based on this requirements the minimum transfer speed can be calculated, let's
consider that a header will be set for every 252 samples (42 for each channel) and
it has 4 bytes (3 of identification - to assure it is the header and not some
data - and a counter). Previewing the worst case, each sample is 2 bytes long.
The transfer rate given by \autoref{usb-generic-rate-equation}.

\begin{equation}
  \label{usb-generic-rate-equation}
  transfer\ rate = \left( \frac{channels * \frac{bytes}{sample} * \frac{samples}{package} + header\ size}{\frac{samples}{package}}  \right ) * f_{s} [B/s]
\end{equation}

Considering that $f_{s}$ has to be somewhere between 10 kHz and 50 kHz the transfer
rate the numerical result is given by \autoref{usb-numerical-rate-equation}.

\begin{equation}
  \label{usb-numerical-rate-equation}
  transfer\ rate = \left( \frac{6 * 2 * 252 + 4}{252}  \right ) * f_{s} = 12.0159 * f_{s} = 120.159 - 600.794 [kB/s]
\end{equation}

USB transfer speed is usually refered in Mbps, which gives us a range between
961.27 and 4806.34 Mbps. This is too high for serial communication (typical max of
1 Mbps) so we need to add a requirement for raw USB support, which allows bulk
transfer that can a have transfer rate up to 12Mbps (USB full-speed standard).\\
We still need to choose or exact sample rate ($f_{s}$), but first let's select which
hardware will be used.

\section{Microcontroller Selection}
There are too many microcontrollers that fit our requirements, but the most popular
and cheap one is clearly the ARM from ST called STM32F103C8T6, and that is why it
was selected. \\
It has eight 12 bits ADC inputs (with 2 parallel channels), DMA for the ADCs and USB full-speed support. It's
also relatively fast (72 MHz clock, 32 bits architecture). All this for under 2 U\$D
in a developing board from China (the actual $\mu$C is under 0.2 U\$). \\

\section{Program}
