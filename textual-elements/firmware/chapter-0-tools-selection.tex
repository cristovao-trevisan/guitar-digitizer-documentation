\chapter{Tools Selection}
\label{tools-selection}
The firmware is basically an analog sampler, all it has to do is sample six analog
channels, add a header (to identify the beginning and check continuity) and send it
through USB. Lets start by listing the requirements for the hardware.

\section{Requirements}
\begin{itemlist}
  \item Super cheap
  \item 6 analog channels (more precision is better)
  \item High sample rate (at least 10 kHz for each channel, but ideally 44 kHz or more)
  \item Fast USB support, to send the data with headers
\end{itemlist}
Based on this requirements the minimum transfer speed can be calculated, let's
consider that a header will be set for every 252 samples (42 for each channel) and
it has 4 bytes (3 of identification - to assure it is the header and not some
data - and a counter). Previewing the worst case, each sample is 2 bytes long.
The transfer rate given by \autoref{usb-generic-rate-equation}.

\begin{equation}
  \label{usb-generic-rate-equation}
  transfer\ rate = \left( \frac{channels * \frac{bytes}{sample} * \frac{samples}{package} + header\ size}{\frac{samples}{package}}  \right ) * f_{s} [B/s]
\end{equation}

Considering that $f_{s}$ has to be somewhere between 10 kHz and 50 kHz the transfer
rate the numerical result is given by \autoref{usb-numerical-rate-equation}.

\begin{equation}
  \label{usb-numerical-rate-equation}
  transfer\ rate = \left( \frac{6 * 2 * 252 + 4}{252}  \right ) * f_{s} = 12.0159 * f_{s} = 120.159 - 600.794 [kB/s]
\end{equation}

USB transfer speed is usually referred in Mbps, which gives us a range between
961.27 and 4806.34 Mbps. This is too high for serial communication (typical max of
1 Mbps) so we need to add a requirement for raw USB support, which allows bulk
transfer that can a have transfer rate up to 12Mbps (USB full-speed standard).\\
We still need to choose or exact sample rate ($f_{s}$), but first let's select which
hardware will be used.

\section{Microcontroller Selection}
There are too many microcontrollers that fit our requirements, but the most popular
and cheap one is clearly the ARM from ST called STM32F103C8T6 \cite{STM32F103},
and that is why it was selected. \\
It has eight 12 bits ADC inputs (with 2 parallel channels), DMA for the ADCs and USB full-speed support. It's
also relatively fast (72 MHz clock, 32 bits architecture). All this for under 2 U\$D
in a developing board from China (the actual $\mu$C is under 0.2 U\$). \\

\section{Sample Frequency}
Usage of the ADCs for the selected $\mu$C can be optimized by using continuous sampling
mode in conjunction with DMA \cite[ch. 11]{STM32F103}. In this mode the sample frequency is controlled by a register
that sets the convolution time, which gives more precision the more ADC clock
cycles (longer) it takes to sample.\\
The first variable chosen for this setup is the ADC clock, which is set by dividing
the $\mu$C clock by 2, 4, 6 or 8. The ADC also has a maximum clock of 14 MHz. Taking in
account the  $\mu$C clock of 72 MHz, the highest possible value for the ADC clock is 12 MHz,
which is set by a divider value of 6. \\
The last value to be chosen is the mentioned convolution time ($T_{c}$ calculate as the selected
value plus 12.5 ADC clock cycles), which gives a sample frequency calculate by \autoref{ADC-sampling-time}.

\begin{equation}
  \label{ADC-sampling-time}
  f_{s} = \frac{ADC\ clock}{T_{c} + 12.5}  * \frac{ADCs}{channels} = \frac{12}{T_{c} + 12.5}  * \frac{2}{6}\ [MHz]
\end{equation}
\autoref{ADC-sampling-frequencies} shows the calculated results for each possible
register value (using \autoref{ADC-sampling-time}). Based on it the chosen sampling
frequency is 47.619 kHz.

\begin{table}[htb]
  \caption{ADC Sampling Frequencies}
  \label{ADC-sampling-frequencies}
  \begin{tabular}{c|c|c}
    \textbf{Register Value} & \textbf{Convolution Time [cycles]} & \textbf{Sampling Frequency [kHz]}\\
    \hline
    000 & 1.5 & 285.71 \\
    \hline
    001 & 7.5 & 200 \\
    \hline
    010 & 13.5 & 153.85 \\
    \hline
    011 & 28.5 & 97.56 \\
    \hline
    100 & 41.5 & 74.07 \\
    \hline
    101 & 55.5 & 58.82 \\
    \hline
    110 & 71.5 & 47.62 \\
    \hline
    111 & 239.5 & 15.87 \\
  \end{tabular}
\end{table}
