\section{Implementation}
\label{firmware-implementation}

\subsection{First Attempt}
We first tried to build our firmware from scratch, using the tools given by the
manufacturer, essentially a set of driver abstractions (HAL drivers). The problem
found is that these abstractions are too slow, and don't work when the firmware
uses the hardware close to it's limits (as we do for both transfer and sampling rates).

\subsection{Second Attempt}
In our research we found an amazing open source project called MiniScope \cite{MiniScope},
in which a few options of low budget DIY digital scope (using different $\mu$Cs) are
presented. One of the $\mu$Cs used by MiniScope is the one we selected, so for our
implementation we took it's firmware as a base project. In this project the author
claims to sample and transfer two channels at 461 kHz (but 8 bits only), which is
very close to our needs (we need a little more transfer but much less sampling speed).

\subsection{IDE}
As we take MiniScope as a base project we will be using the same IDE as it, named
CooCox \cite{CooCox}. It has a full set of tools, and it's completely free (no limitations).

\subsection{Modifications}
The base project samples 2 channels at a different speed, bit rate and does not
add any headers to the data. It also has some code to answer a few commands. It
was as simple as setting up the registers for 6 channels, changing the sample size,
placing our already chosen speed (\autoref{firmware-sample-frequency}) and removing
any unused code. \\
The act of changing the sample size was not done by registers. MiniScope was already
sampling with 12 bits, but it was ignoring the least significant ones when filling
the USB buffer. What we have done is to change the bits alignment and putting all
the data received from the ADC to the USB buffer.

\subsection{Testing}
At first an attempt using the OS (Windows at that time, later Ubuntu) default driver
as made. That did not work well, as it is too generic and thus slow. \\
At a second try, a simple libusb \cite{libusb} program was built to test the transfer
rate (calculated in \autoref{firmware-sample-frequency}). The reported result is
almost perfectly the calculated one.

\subsubsection{Repository}
Again, all code is available at GitHub \cite{guitar-digitizer-firmware}.
