\chapter[Conclusion]{Conclusion}
Even if not having robustness to be a commercial product this project is still a big source
of learning and a proof of concept. To build it, knowledge about almost every area
of electronic engineering was needed at some level. Lots of work still need to be put on it,
but it does have potential to evolve into a useful solution for guitar digitization.

As for engineering effort, it required knowledge from passive hardware, to active hardware
(amplifiers), low level software (firmware), digital communication (USB) and up to high
level software (that can be further divided to even more areas). Summarizing, it was
a great way of learning and proofing our proeficiency in electronic engineering.

The area of musical signal processing proved to be as complex as it popularly said to be.
Still with a fair amount of effort made it is possible to use it in a real world application,
even for non-specialists in signal processing, like ourselves.

Finally our project was fun to build and even more fun to use and play with, and achieved
the desired result of showing that it is possible to make guitar digitization cheap and
efficient, even if not yet to that point.