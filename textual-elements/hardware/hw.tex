\chapter{Hardware}
\section{Initial Idea}
The project started with the idea of assembly a system which should capture de guitar sound and change the sound to any another instrument. The system should be competitive with other existent devices.\\
To reach this objective it was verified the features which the project should reach and the methods necessaries to perform the desired actions.\\
\section{Pickup Project} 
After some studies it was decided to assembly a hexaphonic pickup instead buy one already manufactured, which is an expansive device. To reduce the cost of the project it was projected and printed the pickup base on 3D printer as the project showed on figure 1.

\begin{figure}[!htpb]
\centering
\includegraphics[scale=0.5]{textual-elements/hardware/Capt}
\caption{Figure 1}
\end{figure}

After print the pickup base it was assembled the coils, composed by guitar magnets and copper iron (150 revolutions for each magnet), responsible for the transformation of the mechanic vibration of the string to the electric signal which is needed.   

\begin{figure}[!htpb]
\centering
\includegraphics[scale=0.3]{textual-elements/hardware/ima}
\caption{Figure 2}
\end{figure}

With the pickup already assembled it was verified the output signal. This signal was about 0.5 mA and ths value was so weak to send to the Analogic-digital converter present on the microprocessor. With this dungeon it was verified that it is needed a circuit responsible for the signal amplification to be possible to read the signal with quality on the microprocessor. 

\section{Amplifier Circuits}
It was researched some types of amplifiers circuits, it was selected the circuits with single supply, because the project was designed to be the maximum embedded on the guitar, without the needing of other supplies except the guitar battery or the USB connection of the microprocessor.
After some researches it was decided try two circuit types, with two different operational amplifiers. The first circuit uses the Texas Instruments INA 326 Instrumental Amplifier and the second uses the Texa Instruments TLV 4316 Operational Amplifier.

\subsection{INA 326}
To project the circuit using the INA 326 IC, it was researched the datasheet and it was verified one circuit which provides the desired Gain to the electric signal from the pickup respecting the project's initial requests. The circuit is shown on figure 3. 

\begin{figure}[!htpb]
\centering
\includegraphics[scale=0.8]{textual-elements/hardware/Texas}
\caption{Figure 3}
\end{figure}

This circuit amplifies the signal and the gain is obtained by the following equation:

$$G=2*\frac{(R_2||R_2 ')}{R_1}$$

On the project the Resistor $R_0$ and the Capacitor $C_0$ was excluded because it was not relevant on the output to this project. After the research it was developed a schematic model to perform some simulation tests to verify and prove the functionalism of the circuit for the desired application. The schematic circuit it was developed until the version showed at figure 4.

\begin{figure}[!htpb]
\centering
\includegraphics[scale=0.7]{textual-elements/hardware/INA}
\caption{Figure 4}
\end{figure}

This circuit as showed on the image provides a Gain of:

$$G=2*\frac{(1M\Omega||1M\Omega)}{10k\Omega}$$
$$G=2*\frac{500k\Omega}{10k\Omega}$$
$$G=2*50$$
$$G=100$$

The amplifier circuit was projected to each channel. So the circuit needed 6 IC INA 326 to be complete. The complete schematic is showed on figure 5.

\begin{figure}[!htpb]
\centering
\includegraphics[scale=0.45]{textual-elements/hardware/INAsch}
\caption{Figure 5}
\end{figure}