\section{Software Results and Discussions}

\subsection{Measurements}
For the quantification of how well our system works we need some measurements.
A few tests were thought and measured, according to \autoref{measurements}.
For each test line at least 10 measurements were made. The results show that
our system does work but has a lot of room for improvement, the means to achieve
better results will be discussed in the next section.

\begin{table}[htb]
  \begin{center}
    \ABNTEXreducedfont
    \caption[Measurements]{Measurements}
    \label{measurements}
    \begin{tabular}{p{10cm}|c|c}
      \hline
      Test Description & Average Accuracy & Measurements\\
      \hline \hline
      Single string, pluck single note & XX\% & Y\\ \hline
      Single string, 2 notes sequence with a semitone distance & XX\% & Y \\ \hline
      Single string, 2 notes sequence with a 3 tones distance & XX\% & Y \\ \hline
      2 consecutive strings, single note each & XX\% & Y \\ \hline
      2 consecutive strings, 2 notes sequence with a semitone distance & XX\% & Y \\ \hline
      2 consecutive strings, 2 notes sequence with a 3 tones distance & XX\% & Y \\ \hline
      G Major Chord & XX\% & Y \\ \hline
      A Major Chord & XX\% & Y \\ \hline
      B Minor Chord & XX\% & Y \\ \hline
      \hline
    \end{tabular}
    \legend{Source: Authors}
  \end{center}
\end{table}

\subsection{Future Improvements}
\subsubsection{Note Detection}
Real time note detection proved not to be so accurate. Legato (connected) notes
may cause a middle note detection and there are miss-detections (mostly at frequency
transitions). It works well enough for live play of MIDI instruments, but not
for music notation. 

For real world music notation a different solution is needed: record the signals
and post-process them. This way it is possible to use a more detailed analysis
of each signal and thus get very good results, as there is much more computing
power available when not being limited by real-time processing.

\subsubsection{Performance}
Though real time analysis work, a few limitations were detected. Buffering can only
process each data point one time (can't use a sliding buffer that recycles data).
This means that we are close to the processing limits, due to two causes: single
core processing and slow algorithms. 

One step of the solution is to use multi-core processing for the pitch detection,
which can be done using Node.js support for it. 

As already stated (\autoref{pitch-detection-implementation}) the current implementation
uses quadratic growth algorithms, while it can be improved to logarithmic
growth. Fixing this will improve the performance to a point where multi-core processing
won't be even necessary.

\subsubsection{Algorithm Improvements}
A few algorithm modifications may also improve the project's results. Most noticeable
would be the above mentioned buffering to process each input multiple times, with a
sliding window - which needs performance improvements first.

Another change necessary would be to remove co-channel interference in a more fashioned
way other than amplitude check, making it easier for the pitch detection algorithm to work.

\subsubsection{New Features}
A main feature not yet implemented is also necessary: vibrato detection. Vibrato is the
note's small variation of frequency without changing to another note. This is widely used
by musicians and needs to be implemented, but has not been considered yet.

As already mentioned recording will be a breaking change to the system. The ability to have
more processing power due to not run in real time can greatly improve the system accuracy.
It also makes possible to look ahead of time to check with more precision a single moment
(as notes are frequencies time is an important matter).
