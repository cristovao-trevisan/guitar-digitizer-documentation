\section{Software Results and Discussions}

\subsection{Results}
TODO: measurements and analysis

\subsection{Future Improvements}
\subsubsection{Note Detection}
Real time note detection proved not to be so accurate. Legato (connected) notes
may cause a middle note detection and there are miss-detections (mostly at frequency
transitions). It works well enough for live play of MIDI instruments, but not
for music notation. \\
For real world music notation a different solution is needed: record the signals
and post-process them. This way it is possible to use a more detailed analysis
of each signal and thus get very good results, as there is much more computing
power available when not being limited by real-time processing.

\subsubsection{Performance}
Though real time analysis work, a few limitations were detected. Buffering can only
process each data point one time (can't use a sliding buffer that recycles data).
This means that we are close to the processing limits, due to two causes: single
core processing and slow algorithms. \\
One step of the solution is to use multi-core processing for the pitch detection,
which can be done using Node.js support for it. \\
As already stated (\autoref{pitch-detection-implementation}) the current implementation
uses quadratic growth algorithms, while it can be improved to logarithmic
growth. Fixing this will improve the performance to a point where multi-core processing
won't be even necessary.
