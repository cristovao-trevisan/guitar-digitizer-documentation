% ----------------------------------------------------------
% summarys
% ----------------------------------------------------------
\setlength{\absparsep}{18pt} % adjust the spacing of paragraphs

% summary in english
\begin{summary}[Abstract]
  This is the english abstract.

  \textbf{Key-words}: latex. abntex. text editoration.
\end{summary}

% summary in português
\begin{summary}[Resumo]
  \begin{otherlanguage*}{portuguese}
    Segundo a \citeonline[3.1-3.2]{NBR6028:2003}, o summary deve ressaltar o
    objetivo, o método, os resultados e as conclusões do documento. A ordem e a extensão
    destes itens dependem do tipo de summary (informativo ou indicativo) e do
    tratamento que cada item recebe no documento original. O summary deve ser
    precedido da referência do documento, com exceção do summary inserido no
    próprio documento. (\ldots) As palavras-chave devem figurar logo abaixo do
    summary, antecedidas da expressão Palavras-chave:, separadas entre si por
    ponto e finalizadas também por ponto.

    \textbf{Palavras-chaves}: latex. abntex. editoração de texto.
  \end{otherlanguage*}

\end{summary}

% ----------------------------- NOT COMPILED -----------------------------------
\iffalse
% summary in francês
\begin{summary}[Résumé]
 \begin{otherlanguage*}{french}
    Il s'agit d'un résumé en français.

   \textbf{Mots-clés}: latex. abntex. publication de textes.
 \end{otherlanguage*}
\end{summary}

% summary in espanhol
\begin{summary}[summaryn]
 \begin{otherlanguage*}{spanish}
   Este es el summaryn en español.

   \textbf{Palabras clave}: latex. abntex. publicación de textos.
 \end{otherlanguage*}
\end{summary}
\fi
