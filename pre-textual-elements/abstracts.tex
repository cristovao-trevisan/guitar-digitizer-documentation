\setlength{\absparsep}{18pt} % adjust the spacing of paragraphs

% summary in english
\begin{summary}[Abstract]
  Guitar are one of most popular instruments today, but there is one big disadvantage
  to use it: there is no good and affordable way to digitize it's music. The
  biggest problem with this is the cost to annotate music, as it needs to be done
  by manually. This project tries to build one such system, starting from passive
  hardware (hexaphonic pickup) to modern signal processing (pitch detection),
  attempting to produce a cheap and effective equipment for guitar music annotation
  by means of generating MIDI format data.


  \textbf{Key-words}: guitar. digitizer. MIDI. pitch. detection. hexaphonic.
\end{summary}

% summary in portuguese
\begin{summary}[Resumo]
  \begin{otherlanguage*}{portuguese}
    Violões e guitarras estão entre os instrumentos mais populares da atualidade,
    mas existe uma grande desvantagem em os utilizar: não há um meio barato e eficaz
    para digitalizar sua música. O grande problema com isso é o alto custo para
    transcrever partituras, que atualmente é um processo manual. Esse projeto tenta
    construir um sistema com esse propósito, criando desde sensores passivos (captador
    hexafônico) até processamento digital de sinais moderno (detecção de nota),
    visando um produto barato e eficaz para anotação musical através da geração de
    dados no format MIDI.

    \textbf{Key-words}: guitarra. digitalizador. MIDI. nota. detecção. hexafônico.
  \end{otherlanguage*}

\end{summary}

% ----------------------------- NOT COMPILED -----------------------------------
% % summary in french
% \begin{summary}[Résumé]
%  \begin{otherlanguage*}{french}
%     Il s'agit d'un résumé en français.
%
%    \textbf{Mots-clés}: latex. abntex. publication de textes.
%  \end{otherlanguage*}
% \end{summary}
%
% % summary in spanish
% \begin{summary}[summaryn]
%  \begin{otherlanguage*}{spanish}
%    Este es el summaryn en español.
%
%    \textbf{Palabras clave}: latex. abntex. publicación de textos.
%  \end{otherlanguage*}
% \end{summary}
