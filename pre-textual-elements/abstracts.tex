\setlength{\absparsep}{18pt} % adjust the spacing of paragraphs

% summary in english
\begin{summary}[Abstract]
This work presents a complete system for electric guitar signal acquisition and processing, in order to translate
the raw audio signal to the corresponding musical notes. It is important to mention that, while this is
relatively easy to be accomplished in instruments that have separated keys for each note (e.g. piano),
this is not the case for instruments like the guitar – the one addressed in this work – that have a single
output for multiple notes (each string has about 15 notes). The guitar is a harmonic instrument, as it can
play multiple notes simultaneously, making the translation a challenging task. It is difficult because
multiple notes might be played at the same time and event the same note at different strings.The proposed
approach employs a specially designed pickup, developed and constructed by the authors, with dedicated coils for
each string (in contrast to the single coil regular models). Audio amplifiers based on INA326 OPAMPs
condition the coils’ signals, which are further converted to digital by an acquisition module based on an
ARM Cortex M3 microcontroller, model STM32F103C8T6. The digital samples are transferred to a PC
through the USB bus and processed using pitch detection algorithms. The software is based on
contemporary tools such as Node, NW.js, React+Redux and Functional Programming. Special emphasis
has been dedicated to the applied nature of the project, in the sense that the goal was to obtain a
\emph{ready-to-use} prototype, despite the limitations on the accuracy of the final translation. In other
words, the present work does not focus on state-of-the-art signal processing approaches in order to
obtain a near perfect translation. Instead, it prioritizes an effective hardware and software architecture
aiming a completely functional system for real-time guitar signal translation into the corresponding
notes. Two pitch detection algorithms were evaluated: YIN and McLeod. Results show that YIN algorithm
performs better for this application, achieving decent accuracy values in the practical experiments.


  \textbf{Key-words}: Electric guitar, signal acquisition, MIDI, pitch detection, hexaphonic.
\end{summary}

% summary in portuguese
\begin{summary}[Resumo]
 \begin{otherlanguage*}{portuguese}
    Este trabalho apresenta um sistema completo para aquisição e processamento dos sinais de guitarra,
    com o objetivo de converter os sinais analógicos para suas notas correspondentes. Este processo é denominado \emph{transcrição}. 
    É importante destacar que, embora trate-se de um processo relativamente simples para instrumentos
    como o piano, este não é o caso para instrumentos como a guitarra (abordada neste trabalho), que tem
    uma única fonte para múltiplas notas (cada corda com aproximadamente 15 notas). A guitarra é um instrumento
    harmônico. Desta maneira, pode tocar diversas notas simultaneamente, tornando ainda mais complexa a tradução
    musical. A proposta deste projeto é utilizar um captador dedicado, projetado e construído pelos autores,
    com captação dedicada para cada corda (em contraste com os modelos regulares com um único canal - \emph{single coil}).
    Um amplificador baseado no circuito integrado INA326 condiciona os sinais captados pelos enrolamentos, que são
    então convertidos para digital em um microcontrolador STM32F103C8T6,
    um ARM Cortex M3. As amostras digitais são então transferidas para o PC por meio do USB, para o posterior processamento
    por algoritmos de detecção de frequência.  O programa no PC foi desenvolvido com base em tecnologias atuais como Node.js, NW.js, React+Redux e
    programação funcional. Foi dedicada atenção especial à natureza aplicada do projeto, no sentido de obter-se
    um protótipo funcional, apesar das limitações de acurácia no resultado final da transcrição. Em outras palavras, 
    o foco deste trabalho não é o estudo ou a aplicação do estado da arte em transcrição musical. A prioridade é um sistema completamente funcional
    para a conversão dos sinais captados da guitarra em notas em tempo real. Dois algoritmos de detecção
    de frequência foram comparados: YIN e MacLeod. Os resultados mostram que o algoritmo YIN teve melhor
    desempenho para esta aplicação, mostrando acurácia satisfatória nos experimentos práticos realizados.
  
  
    \textbf{Key-words}: Guitarra, aquisição de sinais, MIDI, detecção de nota, hexafônico.
  \end{otherlanguage*}
\end{summary}

% ----------------------------- NOT COMPILED -----------------------------------
% % summary in french
% \begin{summary}[Résumé]
%  \begin{otherlanguage*}{french}
%     Il s'agit d'un résumé en français.
%This work presents a complete system for guitar signal acquirement and processing, in order to translate the raw audio signal to the corresponding musical notes. It is important to mention that, while this is relatively easy to be accomplished in instruments that have separated keys for each note (e.g. piano), this is not the case for instruments like the guitar – the one addressed in this work – that have a single output for multiple notes (each string has about 15 notes). The guitar is a harmonic instrument, as it can play multiple notes simultaneously, making the translation a challenging task. The proposed approach employs a specially designed pickup, developed and constructed by the authors, with dedicated coils for each string (in contrast to the single coil regular models). Audio amplifiers based on INA326 OPAMPs condition the coils’ signals, which are further converted to digital by an acquisition module based on an ARM Cortex M3 microcontroller, model STM32F103C8T6. The digital samples are transferred to a PC through the USB bus and processed using pitch detection algorithms. The software is based on contemporary tools such as Node, NW.js, React+Redux and Functional Programming. Special emphasis has been dedicated to the applied nature of the project, in the sense that the goal was to obtain a \emph{ready-to-use} prototype, despite the limitations on the accuracy of the final translation. In other words, the present work does not focus on state-of-the-art signal processing approaches in order to obtain a near perfect translation. Instead, it prioritizes an effective hardware and software architecture aiming a completely functional system for real-time guitar signal translation into the corresponding notes. Two pitch detection algorithms were evaluated: YIN and McLeod. Results show that YIN algorithm performs better for this application, achieving decent accuracy values in the practical experiments.

%    \textbf{Mots-clés}: latex. abntex. publication de textes.
%  \end{otherlanguage*}
% \end{summary}
%
% % summary in spanish
% \begin{summary}[summaryn]
%  \begin{otherlanguage*}{spanish}
%    Este es el summaryn en español.
%
%    \textbf{Palabras clave}: latex. abntex. publicación de textos.
%  \end{otherlanguage*}
% \end{summary}
